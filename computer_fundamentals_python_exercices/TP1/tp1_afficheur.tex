% Format   : LaTeX file
% A propos : TD L2-1 
%%%%%%%%%%%%%%%%%%%%%%%%%%%%%%%%%%%%%%%%%%%%%%%%%%%%%%%%%%%%%%%%%%%%%%%%%%%%%%%

\documentclass[11pt]{article}

\usepackage[utf8x]{inputenc}
\usepackage[T1]{fontenc}
\usepackage[french]{babel}
\usepackage{amsmath, amsthm, amsfonts, amssymb}
\usepackage{moreverb}
\usepackage{comment}
\usepackage{multicol}

\newcommand{\Mmae}[1]{{\ttfamily #1}}


\newcounter{cptexercice}
\setcounter{cptexercice}{1}
\newcommand{\exercice}[1]{\vspace{3mm} \noindent {\bf Exercice \thecptexercice.} \addtocounter{cptexercice}{1} {\emph{#1}} \par}
\pagestyle{empty}

\addtolength{\textwidth}{6cm}
\addtolength{\textheight}{7cm}
\addtolength{\voffset}{-3cm}
\addtolength{\hoffset}{-3.2cm}

\begin{document}

\begin{center}
{\LARGE
{\bf Informatique Fondamentale} \\ {\small L1 2021-2022} \\ Travaux Pratiques 1  - Afficheur sept segments}
\end{center}
\noindent\rule{\linewidth}{0.5pt}
%\Mmae{gcc mon\_fichier.c -std=c11 -Wall -Wextra -o mon\_programme }\par\noindent
%Les exercices marqués d'une étoile sont optionnels et à faire dans un second temps.\par
%\noindent\rule{\linewidth}{0.5pt}


On veut concevoir  un afficheur à sept segments.

\begin{enumerate}
   \item Dans un dossier vierge (appelons-le \Mmae{tp1/}) télécharger les fichiers \Mmae{afficheur.py} et \Mmae{expression}.
   \item Avec un terminal assez grand pour couvrir tout l'écran et après s'être placé dans le dossier \Mmae{tp1/},  lancer la commande suivante \Mmae{\$ python afficheur.py}. Si tout se passe bien vous devriez obtenir une sortie semblable à  :
      \begin{verbatim}
cd tp1/
....@......$python afficheur.py
                                                                                    
  0        1        2        3        4        5        6        7        8        9        
                                                                                        
 ----     ----     ----     ----     ----     ----     ----     ----     ----     ----         
     |   |    |        |   |        |    |   |    |        |   |             |   |             
     |   |    |        |   |        |    |   |    |        |   |             |   |             
 ----     ----     ----     ----     ----     ----     ----     ----     ----     ----         
         |             |   |                 |                 |                 |    |        
         |             |   |                 |                 |                 |    |        
          ----              ----     ----     ----              ----              ----         
                                                                     
      \end{verbatim}
      {\bf Explication :} exécuté, le programme python contenu dans  \Mmae{afficheur.py} utilise les expressions écrites dans le fichier \Mmae{expression}
      pour simuler un afficheur à sept segments. 
      L'afficheur utilise, pour chaque segment, une expression booléenne qui dépend de l'écriture binaire du nombre.
      Exemple : l'expression \Mmae{sgh = b0 et b2} signifie que le segment vertical en haut à
      gauche (SGH : segment haut gauche) ne s'allume que si les bits zero et deux du nombre sont à 1. 
      \par

      Pour chaque nombre de 0 à 9, le programme calcul les segments qui 
      doivent être allumés à partir de l'écriture binaire du nombre
      en utilisant les expressions et affiche le resultat. 
      \par
      Pour un nombre écrit en binaire, l'ordre des bits est celui ci : $b_4b_2b_1b_0$.

   \item Modifier le contenu du fichier \Mmae{expression} afin d'obtenir un 
      afficheur correcte, ressemblant à ça :
      \begin{verbatim}
                                                                                             
  0        1        2        3        4        5        6        7        8        9         
                                                                                             
 ----              ----     ----              ----     ----     ----     ----     ----       
|    |        |        |        |   |    |   |        |             |   |    |   |    |      
|    |        |        |        |   |    |   |        |             |   |    |   |    |      
                   ----     ----     ----     ----     ----              ----     ----       
|    |        |   |             |        |        |   |    |        |   |    |        |      
|    |        |   |             |        |        |   |    |        |   |    |        |      
 ----              ----     ----              ----     ----              ----     ----       
                                                                     
      \end{verbatim}

\end{enumerate}


\end{document}
